%%
%% Automatically generated file from DocOnce source
%% (https://github.com/hplgit/doconce/)
%%
%%
% #ifdef PTEX2TEX_EXPLANATION
%%
%% The file follows the ptex2tex extended LaTeX format, see
%% ptex2tex: https://code.google.com/p/ptex2tex/
%%
%% Run
%%      ptex2tex myfile
%% or
%%      doconce ptex2tex myfile
%%
%% to turn myfile.p.tex into an ordinary LaTeX file myfile.tex.
%% (The ptex2tex program: https://code.google.com/p/ptex2tex)
%% Many preprocess options can be added to ptex2tex or doconce ptex2tex
%%
%%      ptex2tex -DMINTED myfile
%%      doconce ptex2tex myfile envir=minted
%%
%% ptex2tex will typeset code environments according to a global or local
%% .ptex2tex.cfg configure file. doconce ptex2tex will typeset code
%% according to options on the command line (just type doconce ptex2tex to
%% see examples). If doconce ptex2tex has envir=minted, it enables the
%% minted style without needing -DMINTED.
% #endif

% #define PREAMBLE

% #ifdef PREAMBLE
%-------------------- begin preamble ----------------------

\documentclass[%
oneside,                 % oneside: electronic viewing, twoside: printing
final,                   % draft: marks overfull hboxes, figures with paths
10pt]{article}

\listfiles               %  print all files needed to compile this document

\usepackage{relsize,makeidx,color,setspace,amsmath,amsfonts,amssymb}
\usepackage[table]{xcolor}
\usepackage{bm,ltablex,microtype}

\usepackage[pdftex]{graphicx}

\usepackage{ptex2tex}
% #ifdef MINTED
\usepackage{minted}
\usemintedstyle{default}
% #endif

\usepackage[T1]{fontenc}
%\usepackage[latin1]{inputenc}
\usepackage{ucs}
\usepackage[utf8x]{inputenc}

\usepackage{lmodern}         % Latin Modern fonts derived from Computer Modern

% Hyperlinks in PDF:
\definecolor{linkcolor}{rgb}{0,0,0.4}
\usepackage{hyperref}
\hypersetup{
    breaklinks=true,
    colorlinks=true,
    linkcolor=linkcolor,
    urlcolor=linkcolor,
    citecolor=black,
    filecolor=black,
    %filecolor=blue,
    pdfmenubar=true,
    pdftoolbar=true,
    bookmarksdepth=3   % Uncomment (and tweak) for PDF bookmarks with more levels than the TOC
    }
%\hyperbaseurl{}   % hyperlinks are relative to this root

\setcounter{tocdepth}{2}  % levels in table of contents

% --- fancyhdr package for fancy headers ---
\usepackage{fancyhdr}
\fancyhf{} % sets both header and footer to nothing
\renewcommand{\headrulewidth}{0pt}
\fancyfoot[LE,RO]{\thepage}
% Ensure copyright on titlepage (article style) and chapter pages (book style)
\fancypagestyle{plain}{
  \fancyhf{}
  \fancyfoot[C]{{\footnotesize \copyright\ 1999-2021, Morten Hjorth-Jensen. Released under CC Attribution-NonCommercial 4.0 license}}
%  \renewcommand{\footrulewidth}{0mm}
  \renewcommand{\headrulewidth}{0mm}
}
% Ensure copyright on titlepages with \thispagestyle{empty}
\fancypagestyle{empty}{
  \fancyhf{}
  \fancyfoot[C]{{\footnotesize \copyright\ 1999-2021, Morten Hjorth-Jensen. Released under CC Attribution-NonCommercial 4.0 license}}
  \renewcommand{\footrulewidth}{0mm}
  \renewcommand{\headrulewidth}{0mm}
}

\pagestyle{fancy}


\usepackage[framemethod=TikZ]{mdframed}

% --- begin definitions of admonition environments ---

% --- end of definitions of admonition environments ---

% prevent orhpans and widows
\clubpenalty = 10000
\widowpenalty = 10000

% --- end of standard preamble for documents ---


% insert custom LaTeX commands...

\raggedbottom
\makeindex
\usepackage[totoc]{idxlayout}   % for index in the toc
\usepackage[nottoc]{tocbibind}  % for references/bibliography in the toc

%-------------------- end preamble ----------------------

\begin{document}

% matching end for #ifdef PREAMBLE
% #endif

\newcommand{\exercisesection}[1]{\subsection*{#1}}


% ------------------- main content ----------------------



% ----------------- title -------------------------

\thispagestyle{empty}

\begin{center}
{\LARGE\bf
\begin{spacing}{1.25}
Data Analysis and Machine Learning: Logistic Regression and Gradient Methods
\end{spacing}
}
\end{center}

% ----------------- author(s) -------------------------

\begin{center}
{\bf Morten Hjorth-Jensen${}^{1, 2}$} \\ [0mm]
\end{center}

\begin{center}
% List of all institutions:
\centerline{{\small ${}^1$Department of Physics and Center for Computing in Science Education, University of Oslo, Norway}}
\centerline{{\small ${}^2$Department of Physics and Astronomy and Facility for Rare Ion Beams and National Superconducting Cyclotron Laboratory, Michigan State University, USA}}
\end{center}
    
% ----------------- end author(s) -------------------------

% --- begin date ---
\begin{center}
Feb 14, 2021
\end{center}
% --- end date ---

\vspace{1cm}


% !split 
\subsection{Logistic Regression}

In linear regression our main interest was centered on learning the
coefficients of a functional fit (say a polynomial) in order to be
able to predict the response of a continuous variable on some unseen
data. The fit to the continuous variable $y_i$ is based on some
independent variables $\hat{x}_i$. Linear regression resulted in
analytical expressions for standard ordinary Least Squares or Ridge
regression (in terms of matrices to invert) for several quantities,
ranging from the variance and thereby the confidence intervals of the
parameters $\hat{\beta}$ to the mean squared error. If we can invert
the product of the design matrices, linear regression gives then a
simple recipe for fitting our data.

% !split 
\subsection{Classification problems}


Classification problems, however, are concerned with outcomes taking
the form of discrete variables (i.e.~categories). We may for example,
on the basis of DNA sequencing for a number of patients, like to find
out which mutations are important for a certain disease; or based on
scans of various patients' brains, figure out if there is a tumor or
not; or given a specific physical system, we'd like to identify its
state, say whether it is an ordered or disordered system (typical
situation in solid state physics); or classify the status of a
patient, whether she/he has a stroke or not and many other similar
situations.

The most common situation we encounter when we apply logistic
regression is that of two possible outcomes, normally denoted as a
binary outcome, true or false, positive or negative, success or
failure etc.

% !split
\subsection{Optimization and Deep learning}

Logistic regression will also serve as our stepping stone towards
neural network algorithms and supervised deep learning. For logistic
learning, the minimization of the cost function leads to a non-linear
equation in the parameters $\hat{\beta}$. The optimization of the
problem calls therefore for minimization algorithms. This forms the
bottle neck of all machine learning algorithms, namely how to find
reliable minima of a multi-variable function. This leads us to the
family of gradient descent methods. The latter are the working horses
of basically all modern machine learning algorithms.

We note also that many of the topics discussed here on logistic 
regression are also commonly used in modern supervised Deep Learning
models, as we will see later.


% !split 
\subsection{Basics}

We consider the case where the dependent variables, also called the
responses or the outcomes, $y_i$ are discrete and only take values
from $k=0,\dots,K-1$ (i.e.~$K$ classes).

The goal is to predict the
output classes from the design matrix $\hat{X}\in\mathbb{R}^{n\times p}$
made of $n$ samples, each of which carries $p$ features or predictors. The
primary goal is to identify the classes to which new unseen samples
belong.

Let us specialize to the case of two classes only, with outputs
$y_i=0$ and $y_i=1$. Our outcomes could represent the status of a
credit card user that could default or not on her/his credit card
debt. That is


\[
y_i = \begin{bmatrix} 0 & \mathrm{no}\\  1 & \mathrm{yes} \end{bmatrix}.
\]



% !split
\subsection{Linear classifier}

Before moving to the logistic model, let us try to use our linear
regression model to classify these two outcomes. We could for example
fit a linear model to the default case if $y_i > 0.5$ and the no
default case $y_i \leq 0.5$.

We would then have our 
weighted linear combination, namely 
\begin{equation}
\hat{y} = \hat{X}^T\hat{\beta} +  \hat{\epsilon},
\end{equation}
where $\hat{y}$ is a vector representing the possible outcomes, $\hat{X}$ is our
$n\times p$ design matrix and $\hat{\beta}$ represents our estimators/predictors.

% !split
\subsection{Some selected properties}

The main problem with our function is that it takes values on the
entire real axis. In the case of logistic regression, however, the
labels $y_i$ are discrete variables. A typical example is the credit
card data discussed below here, where we can set the state of
defaulting the debt to $y_i=1$ and not to $y_i=0$ for one the persons
in the data set (see the full example below).

One simple way to get a discrete output is to have sign
functions that map the output of a linear regressor to values $\{0,1\}$,
$f(s_i)=sign(s_i)=1$ if $s_i\ge 0$ and 0 if otherwise. 
We will encounter this model in our first demonstration of neural networks. Historically it is called the ``perceptron" model in the machine learning
literature. This model is extremely simple. However, in many cases it is more
favorable to use a ``soft" classifier that outputs
the probability of a given category. This leads us to the logistic function.

% !split
\subsection{Simple example}

The following example on data for coronary heart disease (CHD) as function of age may serve as an illustration. In the code here we read and plot whether a person has had CHD (output = 1) or not (output = 0). This ouput  is plotted the person's against age. Clearly, the figure shows that attempting to make a standard linear regression fit may not be very meaningful.

\bpycod
# Common imports
import os
import numpy as np
import pandas as pd
import matplotlib.pyplot as plt
from sklearn.linear_model import LinearRegression, Ridge, Lasso
from sklearn.model_selection import train_test_split
from sklearn.utils import resample
from sklearn.metrics import mean_squared_error
from IPython.display import display
from pylab import plt, mpl
plt.style.use('seaborn')
mpl.rcParams['font.family'] = 'serif'

# Where to save the figures and data files
PROJECT_ROOT_DIR = "Results"
FIGURE_ID = "Results/FigureFiles"
DATA_ID = "DataFiles/"

if not os.path.exists(PROJECT_ROOT_DIR):
    os.mkdir(PROJECT_ROOT_DIR)

if not os.path.exists(FIGURE_ID):
    os.makedirs(FIGURE_ID)

if not os.path.exists(DATA_ID):
    os.makedirs(DATA_ID)

def image_path(fig_id):
    return os.path.join(FIGURE_ID, fig_id)

def data_path(dat_id):
    return os.path.join(DATA_ID, dat_id)

def save_fig(fig_id):
    plt.savefig(image_path(fig_id) + ".png", format='png')

infile = open(data_path("chddata.csv"),'r')

# Read the chd data as  csv file and organize the data into arrays with age group, age, and chd
chd = pd.read_csv(infile, names=('ID', 'Age', 'Agegroup', 'CHD'))
chd.columns = ['ID', 'Age', 'Agegroup', 'CHD']
output = chd['CHD']
age = chd['Age']
agegroup = chd['Agegroup']
numberID  = chd['ID'] 
display(chd)

plt.scatter(age, output, marker='o')
plt.axis([18,70.0,-0.1, 1.2])
plt.xlabel(r'Age')
plt.ylabel(r'CHD')
plt.title(r'Age distribution and Coronary heart disease')
plt.show()
\epycod

% !split
\subsection{Plotting the mean value for each group}

What we could attempt however is to plot the mean value for each group.

\bpycod
agegroupmean = np.array([0.1, 0.133, 0.250, 0.333, 0.462, 0.625, 0.765, 0.800])
group = np.array([1, 2, 3, 4, 5, 6, 7, 8])
plt.plot(group, agegroupmean, "r-")
plt.axis([0,9,0, 1.0])
plt.xlabel(r'Age group')
plt.ylabel(r'CHD mean values')
plt.title(r'Mean values for each age group')
plt.show()
\epycod

We are now trying to find a function $f(y\vert x)$, that is a function which gives us an expected value for the output $y$ with a given input $x$.
In standard linear regression with a linear dependence on $x$, we would write this in terms of our model
\[
f(y_i\vert x_i)=\beta_0+\beta_1 x_i.
\]

This expression implies however that $f(y_i\vert x_i)$ could take any
value from minus infinity to plus infinity. If we however let
$f(y\vert y)$ be represented by the mean value, the above example
shows us that we can constrain the function to take values between
zero and one, that is we have $0 \le f(y_i\vert x_i) \le 1$. Looking
at our last curve we see also that it has an S-shaped form. This leads
us to a very popular model for the function $f$, namely the so-called
Sigmoid function or logistic model. We will consider this function as
representing the probability for finding a value of $y_i$ with a given
$x_i$.

% !split
\subsection{The logistic function}

Another widely studied model, is the so-called 
perceptron model, which is an example of a ``hard classification'' model. We
will encounter this model when we discuss neural networks as
well. Each datapoint is deterministically assigned to a category (i.e
$y_i=0$ or $y_i=1$). In many cases, and the coronary heart disease data forms one of many such examples, it is favorable to have a ``soft''
classifier that outputs the probability of a given category rather
than a single value. For example, given $x_i$, the classifier
outputs the probability of being in a category $k$.  Logistic regression
is the most common example of a so-called soft classifier. In logistic
regression, the probability that a data point $x_i$
belongs to a category $y_i=\{0,1\}$ is given by the so-called logit function (or Sigmoid) which is meant to represent the likelihood for a given event, 
\[
p(t) = \frac{1}{1+\mathrm \exp{-t}}=\frac{\exp{t}}{1+\mathrm \exp{t}}.
\]
Note that $1-p(t)= p(-t)$.

% !split
\subsection{Examples of likelihood functions used in logistic regression and nueral networks}


The following code plots the logistic function, the step function and other functions we will encounter from here and on.


\bpycod
"""The sigmoid function (or the logistic curve) is a
function that takes any real number, z, and outputs a number (0,1).
It is useful in neural networks for assigning weights on a relative scale.
The value z is the weighted sum of parameters involved in the learning algorithm."""

import numpy
import matplotlib.pyplot as plt
import math as mt

z = numpy.arange(-5, 5, .1)
sigma_fn = numpy.vectorize(lambda z: 1/(1+numpy.exp(-z)))
sigma = sigma_fn(z)

fig = plt.figure()
ax = fig.add_subplot(111)
ax.plot(z, sigma)
ax.set_ylim([-0.1, 1.1])
ax.set_xlim([-5,5])
ax.grid(True)
ax.set_xlabel('z')
ax.set_title('sigmoid function')

plt.show()

"""Step Function"""
z = numpy.arange(-5, 5, .02)
step_fn = numpy.vectorize(lambda z: 1.0 if z >= 0.0 else 0.0)
step = step_fn(z)

fig = plt.figure()
ax = fig.add_subplot(111)
ax.plot(z, step)
ax.set_ylim([-0.5, 1.5])
ax.set_xlim([-5,5])
ax.grid(True)
ax.set_xlabel('z')
ax.set_title('step function')

plt.show()

"""tanh Function"""
z = numpy.arange(-2*mt.pi, 2*mt.pi, 0.1)
t = numpy.tanh(z)

fig = plt.figure()
ax = fig.add_subplot(111)
ax.plot(z, t)
ax.set_ylim([-1.0, 1.0])
ax.set_xlim([-2*mt.pi,2*mt.pi])
ax.grid(True)
ax.set_xlabel('z')
ax.set_title('tanh function')

plt.show()
\epycod







% !split
\subsection{Two parameters}

We assume now that we have two classes with $y_i$ either $0$ or $1$. Furthermore we assume also that we have only two parameters $\beta$ in our fitting of the Sigmoid function, that is we define probabilities 
\begin{align*}
p(y_i=1|x_i,\hat{\beta}) &= \frac{\exp{(\beta_0+\beta_1x_i)}}{1+\exp{(\beta_0+\beta_1x_i)}},\nonumber\\
p(y_i=0|x_i,\hat{\beta}) &= 1 - p(y_i=1|x_i,\hat{\beta}),
\end{align*}
where $\hat{\beta}$ are the weights we wish to extract from data, in our case $\beta_0$ and $\beta_1$. 

Note that we used
\[
p(y_i=0\vert x_i, \hat{\beta}) = 1-p(y_i=1\vert x_i, \hat{\beta}).
\]

% !split 
\subsection{Maximum likelihood}

In order to define the total likelihood for all possible outcomes from a  
dataset $\mathcal{D}=\{(y_i,x_i)\}$, with the binary labels
$y_i\in\{0,1\}$ and where the data points are drawn independently, we use the so-called \href{{https://en.wikipedia.org/wiki/Maximum_likelihood_estimation}}{Maximum Likelihood Estimation} (MLE) principle. 
We aim thus at maximizing 
the probability of seeing the observed data. We can then approximate the 
likelihood in terms of the product of the individual probabilities of a specific outcome $y_i$, that is 
\begin{align*}
P(\mathcal{D}|\hat{\beta})& = \prod_{i=1}^n \left[p(y_i=1|x_i,\hat{\beta})\right]^{y_i}\left[1-p(y_i=1|x_i,\hat{\beta}))\right]^{1-y_i}\nonumber \\
\end{align*}
from which we obtain the log-likelihood and our \textbf{cost/loss} function
\[
\mathcal{C}(\hat{\beta}) = \sum_{i=1}^n \left( y_i\log{p(y_i=1|x_i,\hat{\beta})} + (1-y_i)\log\left[1-p(y_i=1|x_i,\hat{\beta}))\right]\right).
\]

% !split
\subsection{The cost function rewritten}

Reordering the logarithms, we can rewrite the \textbf{cost/loss} function as
\[
\mathcal{C}(\hat{\beta}) = \sum_{i=1}^n  \left(y_i(\beta_0+\beta_1x_i) -\log{(1+\exp{(\beta_0+\beta_1x_i)})}\right).
\]

The maximum likelihood estimator is defined as the set of parameters that maximize the log-likelihood where we maximize with respect to $\beta$.
Since the cost (error) function is just the negative log-likelihood, for logistic regression we have that
\[
\mathcal{C}(\hat{\beta})=-\sum_{i=1}^n  \left(y_i(\beta_0+\beta_1x_i) -\log{(1+\exp{(\beta_0+\beta_1x_i)})}\right).
\]
This equation is known in statistics as the \textbf{cross entropy}. Finally, we note that just as in linear regression, 
in practice we often supplement the cross-entropy with additional regularization terms, usually $L_1$ and $L_2$ regularization as we did for Ridge and Lasso regression.

% !split
\subsection{Minimizing the cross entropy}

The cross entropy is a convex function of the weights $\hat{\beta}$ and,
therefore, any local minimizer is a global minimizer. 


Minimizing this
cost function with respect to the two parameters $\beta_0$ and $\beta_1$ we obtain

\[
\frac{\partial \mathcal{C}(\hat{\beta})}{\partial \beta_0} = -\sum_{i=1}^n  \left(y_i -\frac{\exp{(\beta_0+\beta_1x_i)}}{1+\exp{(\beta_0+\beta_1x_i)}}\right),
\]
and 
\[
\frac{\partial \mathcal{C}(\hat{\beta})}{\partial \beta_1} = -\sum_{i=1}^n  \left(y_ix_i -x_i\frac{\exp{(\beta_0+\beta_1x_i)}}{1+\exp{(\beta_0+\beta_1x_i)}}\right).
\]

% !split
\subsection{A more compact expression}

Let us now define a vector $\hat{y}$ with $n$ elements $y_i$, an
$n\times p$ matrix $\hat{X}$ which contains the $x_i$ values and a
vector $\hat{p}$ of fitted probabilities $p(y_i\vert x_i,\hat{\beta})$. We can rewrite in a more compact form the first
derivative of cost function as

\[
\frac{\partial \mathcal{C}(\hat{\beta})}{\partial \hat{\beta}} = -\hat{X}^T\left(\hat{y}-\hat{p}\right). 
\]

If we in addition define a diagonal matrix $\hat{W}$ with elements 
$p(y_i\vert x_i,\hat{\beta})(1-p(y_i\vert x_i,\hat{\beta})$, we can obtain a compact expression of the second derivative as 

\[
\frac{\partial^2 \mathcal{C}(\hat{\beta})}{\partial \hat{\beta}\partial \hat{\beta}^T} = \hat{X}^T\hat{W}\hat{X}. 
\]

% !split
\subsection{Extending to more predictors}

Within a binary classification problem, we can easily expand our model to include multiple predictors. Our ratio between likelihoods is then with $p$ predictors
\[
\log{ \frac{p(\hat{\beta}\hat{x})}{1-p(\hat{\beta}\hat{x})}} = \beta_0+\beta_1x_1+\beta_2x_2+\dots+\beta_px_p.
\]
Here we defined $\hat{x}=[1,x_1,x_2,\dots,x_p]$ and $\hat{\beta}=[\beta_0, \beta_1, \dots, \beta_p]$ leading to
\[
p(\hat{\beta}\hat{x})=\frac{ \exp{(\beta_0+\beta_1x_1+\beta_2x_2+\dots+\beta_px_p)}}{1+\exp{(\beta_0+\beta_1x_1+\beta_2x_2+\dots+\beta_px_p)}}.
\]

% !split
\subsection{Including more classes}

Till now we have mainly focused on two classes, the so-called binary
system. Suppose we wish to extend to $K$ classes.  Let us for the sake
of simplicity assume we have only two predictors. We have then following model

\[
\log{\frac{p(C=1\vert x)}{p(K\vert x)}} = \beta_{10}+\beta_{11}x_1,
\]
and 
\[
\log{\frac{p(C=2\vert x)}{p(K\vert x)}} = \beta_{20}+\beta_{21}x_1,
\]
and so on till the class $C=K-1$ class
\[
\log{\frac{p(C=K-1\vert x)}{p(K\vert x)}} = \beta_{(K-1)0}+\beta_{(K-1)1}x_1,
\]

and the model is specified in term of $K-1$ so-called log-odds or
\textbf{logit} transformations.


% !split
\subsection{More classes}

In our discussion of neural networks we will encounter the above again
in terms of a slightly modified function, the so-called \textbf{Softmax} function.

The softmax function is used in various multiclass classification
methods, such as multinomial logistic regression (also known as
softmax regression), multiclass linear discriminant analysis, naive
Bayes classifiers, and artificial neural networks.  Specifically, in
multinomial logistic regression and linear discriminant analysis, the
input to the function is the result of $K$ distinct linear functions,
and the predicted probability for the $k$-th class given a sample
vector $\hat{x}$ and a weighting vector $\hat{\beta}$ is (with two
predictors):

\[
p(C=k\vert \mathbf {x} )=\frac{\exp{(\beta_{k0}+\beta_{k1}x_1)}}{1+\sum_{l=1}^{K-1}\exp{(\beta_{l0}+\beta_{l1}x_1)}}.
\]
It is easy to extend to more predictors. The final class is 
\[
p(C=K\vert \mathbf {x} )=\frac{1}{1+\sum_{l=1}^{K-1}\exp{(\beta_{l0}+\beta_{l1}x_1)}},
\]

and they sum to one. Our earlier discussions were all specialized to
the case with two classes only. It is easy to see from the above that
what we derived earlier is compatible with these equations.

To find the optimal parameters we would typically use a gradient
descent method.  Newton's method and gradient descent methods are
discussed in the material on optimization
methods below in these slides.




% !split
\subsection{Wisconsin Cancer Data}

We show here how we can use a simple regression case on the breast
cancer data using Logistic regression as our algorithm for
classification.


\bpycod
import matplotlib.pyplot as plt
import numpy as np
from sklearn.model_selection import  train_test_split 
from sklearn.datasets import load_breast_cancer
from sklearn.linear_model import LogisticRegression

# Load the data
cancer = load_breast_cancer()

X_train, X_test, y_train, y_test = train_test_split(cancer.data,cancer.target,random_state=0)
print(X_train.shape)
print(X_test.shape)
# Logistic Regression
logreg = LogisticRegression(solver='lbfgs')
logreg.fit(X_train, y_train)
print("Test set accuracy with Logistic Regression: {:.2f}".format(logreg.score(X_test,y_test)))
#now scale the data
from sklearn.preprocessing import StandardScaler
scaler = StandardScaler()
scaler.fit(X_train)
X_train_scaled = scaler.transform(X_train)
X_test_scaled = scaler.transform(X_test)
# Logistic Regression
logreg.fit(X_train_scaled, y_train)
print("Test set accuracy Logistic Regression with scaled data: {:.2f}".format(logreg.score(X_test_scaled,y_test)))
\epycod

% !split
\subsection{Using the correlation matrix}

In addition to the above scores, we could also study the covariance (and the correlation matrix).
We use \textbf{Pandas} to compute the correlation matrix.
\bpycod
import matplotlib.pyplot as plt
import numpy as np
from sklearn.model_selection import  train_test_split 
from sklearn.datasets import load_breast_cancer
from sklearn.linear_model import LogisticRegression
cancer = load_breast_cancer()
import pandas as pd
# Making a data frame
cancerpd = pd.DataFrame(cancer.data, columns=cancer.feature_names)

fig, axes = plt.subplots(15,2,figsize=(10,20))
malignant = cancer.data[cancer.target == 0]
benign = cancer.data[cancer.target == 1]
ax = axes.ravel()

for i in range(30):
    _, bins = np.histogram(cancer.data[:,i], bins =50)
    ax[i].hist(malignant[:,i], bins = bins, alpha = 0.5)
    ax[i].hist(benign[:,i], bins = bins, alpha = 0.5)
    ax[i].set_title(cancer.feature_names[i])
    ax[i].set_yticks(())
ax[0].set_xlabel("Feature magnitude")
ax[0].set_ylabel("Frequency")
ax[0].legend(["Malignant", "Benign"], loc ="best")
fig.tight_layout()
plt.show()

import seaborn as sns
correlation_matrix = cancerpd.corr().round(1)
# use the heatmap function from seaborn to plot the correlation matrix
# annot = True to print the values inside the square
plt.figure(figsize=(15,8))
sns.heatmap(data=correlation_matrix, annot=True)
plt.show()


\epycod

% !split
\subsection{Discussing the correlation data}

In the above example we note two things. In the first plot we display
the overlap of benign and malignant tumors as functions of the various
features in the Wisconsing breast cancer data set. We see that for
some of the features we can distinguish clearly the benign and
malignant cases while for other features we cannot. This can point to
us which features may be of greater interest when we wish to classify
a benign or not benign tumour.

In the second figure we have computed the so-called correlation
matrix, which in our case with thirty features becomes a $30\times 30$
matrix.

We constructed this matrix using \textbf{pandas} via the statements
\bpycod
cancerpd = pd.DataFrame(cancer.data, columns=cancer.feature_names)
\epycod
and then
\bpycod
correlation_matrix = cancerpd.corr().round(1)
\epycod

Diagonalizing this matrix we can in turn say something about which
features are of relevance and which are not. This leads  us to
the classical Principal Component Analysis (PCA) theorem with
applications. 



% !split
\subsection{Other measures in classification studies: Cancer Data  again}
\bpycod
import matplotlib.pyplot as plt
import numpy as np
from sklearn.model_selection import  train_test_split 
from sklearn.datasets import load_breast_cancer
from sklearn.linear_model import LogisticRegression

# Load the data
cancer = load_breast_cancer()

X_train, X_test, y_train, y_test = train_test_split(cancer.data,cancer.target,random_state=0)
print(X_train.shape)
print(X_test.shape)
# Logistic Regression
logreg = LogisticRegression(solver='lbfgs')
logreg.fit(X_train, y_train)
print("Test set accuracy with Logistic Regression: {:.2f}".format(logreg.score(X_test,y_test)))
#now scale the data
from sklearn.preprocessing import StandardScaler
scaler = StandardScaler()
scaler.fit(X_train)
X_train_scaled = scaler.transform(X_train)
X_test_scaled = scaler.transform(X_test)
# Logistic Regression
logreg.fit(X_train_scaled, y_train)
print("Test set accuracy Logistic Regression with scaled data: {:.2f}".format(logreg.score(X_test_scaled,y_test)))


from sklearn.preprocessing import LabelEncoder
from sklearn.model_selection import cross_validate
#Cross validation
accuracy = cross_validate(logreg,X_test_scaled,y_test,cv=10)['test_score']
print(accuracy)
print("Test set accuracy with Logistic Regression  and scaled data: {:.2f}".format(logreg.score(X_test_scaled,y_test)))


import scikitplot as skplt
y_pred = logreg.predict(X_test_scaled)
skplt.metrics.plot_confusion_matrix(y_test, y_pred, normalize=True)
plt.show()
y_probas = logreg.predict_proba(X_test_scaled)
skplt.metrics.plot_roc(y_test, y_probas)
plt.show()
skplt.metrics.plot_cumulative_gain(y_test, y_probas)
plt.show()

\epycod



% !split
\subsection{Optimization, the central part of any Machine Learning algortithm}

Almost every problem in machine learning and data science starts with
a dataset $X$, a model $g(\beta)$, which is a function of the
parameters $\beta$ and a cost function $C(X, g(\beta))$ that allows
us to judge how well the model $g(\beta)$ explains the observations
$X$. The model is fit by finding the values of $\beta$ that minimize
the cost function. Ideally we would be able to solve for $\beta$
analytically, however this is not possible in general and we must use
some approximative/numerical method to compute the minimum.


% !split
\subsection{Revisiting our Logistic Regression case}

In our discussion on Logistic Regression we studied the 
case of
two classes, with $y_i$ either
$0$ or $1$. Furthermore we assumed also that we have only two
parameters $\beta$ in our fitting, that is we
defined probabilities

\begin{align*}
p(y_i=1|x_i,\bm{\beta}) &= \frac{\exp{(\beta_0+\beta_1x_i)}}{1+\exp{(\beta_0+\beta_1x_i)}},\nonumber\\
p(y_i=0|x_i,\bm{\beta}) &= 1 - p(y_i=1|x_i,\bm{\beta}),
\end{align*}
where $\bm{\beta}$ are the weights we wish to extract from data, in our case $\beta_0$ and $\beta_1$. 

% !split
\subsection{The equations to solve}

Our compact equations used a definition of a vector $\bm{y}$ with $n$
elements $y_i$, an $n\times p$ matrix $\bm{X}$ which contains the
$x_i$ values and a vector $\bm{p}$ of fitted probabilities
$p(y_i\vert x_i,\bm{\beta})$. We rewrote in a more compact form
the first derivative of the cost function as

\[
\frac{\partial \mathcal{C}(\bm{\beta})}{\partial \bm{\beta}} = -\bm{X}^T\left(\bm{y}-\bm{p}\right). 
\]

If we in addition define a diagonal matrix $\bm{W}$ with elements 
$p(y_i\vert x_i,\bm{\beta})(1-p(y_i\vert x_i,\bm{\beta})$, we can obtain a compact expression of the second derivative as 

\[
\frac{\partial^2 \mathcal{C}(\bm{\beta})}{\partial \bm{\beta}\partial \bm{\beta}^T} = \bm{X}^T\bm{W}\bm{X}. 
\]
This defines what is called  the Hessian matrix.

% !split
\subsection{Solving using Newton-Raphson's method}

If we can set up these equations, Newton-Raphson's iterative method is normally the method of choice. It requires however that we can compute in an efficient way the  matrices that define the first and second derivatives. 

Our iterative scheme is then given by

\[
\bm{\beta}^{\mathrm{new}} = \bm{\beta}^{\mathrm{old}}-\left(\frac{\partial^2 \mathcal{C}(\bm{\beta})}{\partial \bm{\beta}\partial \bm{\beta}^T}\right)^{-1}_{\bm{\beta}^{\mathrm{old}}}\times \left(\frac{\partial \mathcal{C}(\bm{\beta})}{\partial \bm{\beta}}\right)_{\bm{\beta}^{\mathrm{old}}},
\]
or in matrix form as

\[
\bm{\beta}^{\mathrm{new}} = \bm{\beta}^{\mathrm{old}}-\left(\bm{X}^T\bm{W}\bm{X} \right)^{-1}\times \left(-\bm{X}^T(\bm{y}-\bm{p}) \right)_{\bm{\beta}^{\mathrm{old}}}.
\]
The right-hand side is computed with the old values of $\beta$. 

If we can compute these matrices, in particular the Hessian, the above is often the easiest method to implement. 


% !split
\subsection{Brief reminder on Newton-Raphson's method}

Let us quickly remind ourselves how we derive the above method.

Perhaps the most celebrated of all one-dimensional root-finding
routines is Newton's method, also called the Newton-Raphson
method. This method  requires the evaluation of both the
function $f$ and its derivative $f'$ at arbitrary points. 
If you can only calculate the derivative
numerically and/or your function is not of the smooth type, we
normally discourage the use of this method.

% !split
\subsection{The equations}

The Newton-Raphson formula consists geometrically of extending the
tangent line at a current point until it crosses zero, then setting
the next guess to the abscissa of that zero-crossing.  The mathematics
behind this method is rather simple. Employing a Taylor expansion for
$x$ sufficiently close to the solution $s$, we have


\[
    f(s)=0=f(x)+(s-x)f'(x)+\frac{(s-x)^2}{2}f''(x) +\dots.
    \label{eq:taylornr}
\]

For small enough values of the function and for well-behaved
functions, the terms beyond linear are unimportant, hence we obtain


\[
   f(x)+(s-x)f'(x)\approx 0,
\]
yielding
\[
   s\approx x-\frac{f(x)}{f'(x)}.
\]

Having in mind an iterative procedure, it is natural to start iterating with
\[
   x_{n+1}=x_n-\frac{f(x_n)}{f'(x_n)}.
\]

% !split
\subsection{Simple geometric interpretation}

The above is Newton-Raphson's method. It has a simple geometric
interpretation, namely $x_{n+1}$ is the point where the tangent from
$(x_n,f(x_n))$ crosses the $x$-axis.  Close to the solution,
Newton-Raphson converges fast to the desired result. However, if we
are far from a root, where the higher-order terms in the series are
important, the Newton-Raphson formula can give grossly inaccurate
results. For instance, the initial guess for the root might be so far
from the true root as to let the search interval include a local
maximum or minimum of the function.  If an iteration places a trial
guess near such a local extremum, so that the first derivative nearly
vanishes, then Newton-Raphson may fail totally


% !split
\subsection{Extending to more than one variable}

Newton's method can be generalized to systems of several non-linear equations
and variables. Consider the case with two equations
\[
   \begin{array}{cc} f_1(x_1,x_2) &=0\\
                     f_2(x_1,x_2) &=0,\end{array}
\]
which we Taylor expand to obtain

\[
   \begin{array}{cc} 0=f_1(x_1+h_1,x_2+h_2)=&f_1(x_1,x_2)+h_1
                     \partial f_1/\partial x_1+h_2
                     \partial f_1/\partial x_2+\dots\\
                     0=f_2(x_1+h_1,x_2+h_2)=&f_2(x_1,x_2)+h_1
                     \partial f_2/\partial x_1+h_2
                     \partial f_2/\partial x_2+\dots
                       \end{array}.
\]
Defining the Jacobian matrix ${\bf \bm{J}}$ we have
\[
 {\bf \bm{J}}=\left( \begin{array}{cc}
                         \partial f_1/\partial x_1  & \partial f_1/\partial x_2 \\
                          \partial f_2/\partial x_1     &\partial f_2/\partial x_2
             \end{array} \right),
\]
we can rephrase Newton's method as
\[
\left(\begin{array}{c} x_1^{n+1} \\ x_2^{n+1} \end{array} \right)=
\left(\begin{array}{c} x_1^{n} \\ x_2^{n} \end{array} \right)+
\left(\begin{array}{c} h_1^{n} \\ h_2^{n} \end{array} \right),
\]
where we have defined
\[
   \left(\begin{array}{c} h_1^{n} \\ h_2^{n} \end{array} \right)=
   -{\bf \bm{J}}^{-1}
   \left(\begin{array}{c} f_1(x_1^{n},x_2^{n}) \\ f_2(x_1^{n},x_2^{n}) \end{array} \right).
\]
We need thus to compute the inverse of the Jacobian matrix and it
is to understand that difficulties  may
arise in case ${\bf \bm{J}}$ is nearly singular.

It is rather straightforward to extend the above scheme to systems of
more than two non-linear equations. In our case, the Jacobian matrix is given by the Hessian that represents the second derivative of cost function. 



% !split
\subsection{Steepest descent}

The basic idea of gradient descent is
that a function $F(\mathbf{x})$, 
$\mathbf{x} \equiv (x_1,\cdots,x_n)$, decreases fastest if one goes from $\bf {x}$ in the
direction of the negative gradient $-\nabla F(\mathbf{x})$.

It can be shown that if 
\[
\mathbf{x}_{k+1} = \mathbf{x}_k - \gamma_k \nabla F(\mathbf{x}_k),
\]
with $\gamma_k > 0$.

For $\gamma_k$ small enough, then $F(\mathbf{x}_{k+1}) \leq
F(\mathbf{x}_k)$. This means that for a sufficiently small $\gamma_k$
we are always moving towards smaller function values, i.e a minimum.

% !split 
\subsection{More on Steepest descent}

The previous observation is the basis of the method of steepest
descent, which is also referred to as just gradient descent (GD). One
starts with an initial guess $\mathbf{x}_0$ for a minimum of $F$ and
computes new approximations according to

\[
\mathbf{x}_{k+1} = \mathbf{x}_k - \gamma_k \nabla F(\mathbf{x}_k), \ \ k \geq 0.
\]

The parameter $\gamma_k$ is often referred to as the step length or
the learning rate within the context of Machine Learning.

% !split 
\subsection{The ideal}

Ideally the sequence $\{\mathbf{x}_k \}_{k=0}$ converges to a global
minimum of the function $F$. In general we do not know if we are in a
global or local minimum. In the special case when $F$ is a convex
function, all local minima are also global minima, so in this case
gradient descent can converge to the global solution. The advantage of
this scheme is that it is conceptually simple and straightforward to
implement. However the method in this form has some severe
limitations:

In machine learing we are often faced with non-convex high dimensional
cost functions with many local minima. Since GD is deterministic we
will get stuck in a local minimum, if the method converges, unless we
have a very good intial guess. This also implies that the scheme is
sensitive to the chosen initial condition.

Note that the gradient is a function of $\mathbf{x} =
(x_1,\cdots,x_n)$ which makes it expensive to compute numerically.


% !split 
\subsection{The sensitiveness of the gradient descent}

The gradient descent method 
is sensitive to the choice of learning rate $\gamma_k$. This is due
to the fact that we are only guaranteed that $F(\mathbf{x}_{k+1}) \leq
F(\mathbf{x}_k)$ for sufficiently small $\gamma_k$. The problem is to
determine an optimal learning rate. If the learning rate is chosen too
small the method will take a long time to converge and if it is too
large we can experience erratic behavior.

Many of these shortcomings can be alleviated by introducing
randomness. One such method is that of Stochastic Gradient Descent
(SGD), see below.


% !split 
\subsection{Convex functions}

Ideally we want our cost/loss function to be convex(concave).

First we give the definition of a convex set: A set $C$ in
$\mathbb{R}^n$ is said to be convex if, for all $x$ and $y$ in $C$ and
all $t \in (0,1)$ , the point $(1 − t)x + ty$ also belongs to
C. Geometrically this means that every point on the line segment
connecting $x$ and $y$ is in $C$ as discussed below.

The convex subsets of $\mathbb{R}$ are the intervals of
$\mathbb{R}$. Examples of convex sets of $\mathbb{R}^2$ are the
regular polygons (triangles, rectangles, pentagons, etc...).

% !split
\subsection{Convex function}

\textbf{Convex function}: Let $X \subset \mathbb{R}^n$ be a convex set. Assume that the function $f: X \rightarrow \mathbb{R}$ is continuous, then $f$ is said to be convex if $$f(tx_1 + (1-t)x_2) \leq tf(x_1) + (1-t)f(x_2) $$ for all $x_1, x_2 \in X$ and for all $t \in [0,1]$. If $\leq$ is replaced with a strict inequaltiy in the definition, we demand $x_1 \neq x_2$ and $t\in(0,1)$ then $f$ is said to be strictly convex. For a single variable function, convexity means that if you draw a straight line connecting $f(x_1)$ and $f(x_2)$, the value of the function on the interval $[x_1,x_2]$ is always below the line as illustrated below.

% !split
\subsection{Conditions on convex functions}

In the following we state first and second-order conditions which
ensures convexity of a function $f$. We write $D_f$ to denote the
domain of $f$, i.e the subset of $R^n$ where $f$ is defined. For more
details and proofs we refer to: \href{{http://stanford.edu/boyd/cvxbook/, 2004}}{S. Boyd and L. Vandenberghe. Convex Optimization. Cambridge University Press}.


% --- begin paragraph admon ---
\paragraph{First order condition.}
Suppose $f$ is differentiable (i.e $\nabla f(x)$ is well defined for
all $x$ in the domain of $f$). Then $f$ is convex if and only if $D_f$
is a convex set and $$f(y) \geq f(x) + \nabla f(x)^T (y-x) $$ holds
for all $x,y \in D_f$. This condition means that for a convex function
the first order Taylor expansion (right hand side above) at any point
a global under estimator of the function. To convince yourself you can
make a drawing of $f(x) = x^2+1$ and draw the tangent line to $f(x)$ and
note that it is always below the graph.
% --- end paragraph admon ---




% --- begin paragraph admon ---
\paragraph{Second order condition.}
Assume that $f$ is twice
differentiable, i.e the Hessian matrix exists at each point in
$D_f$. Then $f$ is convex if and only if $D_f$ is a convex set and its
Hessian is positive semi-definite for all $x\in D_f$. For a
single-variable function this reduces to $f''(x) \geq 0$. Geometrically this means that $f$ has nonnegative curvature
everywhere.
% --- end paragraph admon ---



This condition is particularly useful since it gives us an procedure for determining if the function under consideration is convex, apart from using the definition.

% !split
\subsection{More on convex functions}

The next result is of great importance to us and the reason why we are
going on about convex functions. In machine learning we frequently
have to minimize a loss/cost function in order to find the best
parameters for the model we are considering. 

Ideally we want the
global minimum (for high-dimensional models it is hard to know
if we have local or global minimum). However, if the cost/loss function
is convex the following result provides invaluable information:


% --- begin paragraph admon ---
\paragraph{Any minimum is global for convex functions.}
Consider the problem of finding $x \in \mathbb{R}^n$ such that $f(x)$
is minimal, where $f$ is convex and differentiable. Then, any point
$x^*$ that satisfies $\nabla f(x^*) = 0$ is a global minimum.
% --- end paragraph admon ---



This result means that if we know that the cost/loss function is convex and we are able to find a minimum, we are guaranteed that it is a global minimum.

% !split
\subsection{Some simple problems}

\begin{enumerate}
\item Show that $f(x)=x^2$ is convex for $x \in \mathbb{R}$ using the definition of convexity. Hint: If you re-write the definition, $f$ is convex if the following holds for all $x,y \in D_f$ and any $\lambda \in [0,1]$ $\lambda f(x)+(1-\lambda)f(y)-f(\lambda x + (1-\lambda) y ) \geq 0$.

\item Using the second order condition show that the following functions are convex on the specified domain.
\begin{itemize}

 \item $f(x) = e^x$ is convex for $x \in \mathbb{R}$.

 \item $g(x) = -\ln(x)$ is convex for $x \in (0,\infty)$.

\end{itemize}

\noindent
\item Let $f(x) = x^2$ and $g(x) = e^x$. Show that $f(g(x))$ and $g(f(x))$ is convex for $x \in \mathbb{R}$. Also show that if $f(x)$ is any convex function than $h(x) = e^{f(x)}$ is convex.

\item A norm is any function that satisfy the following properties
\begin{itemize}

 \item $f(\alpha x) = |\alpha| f(x)$ for all $\alpha \in \mathbb{R}$.

 \item $f(x+y) \leq f(x) + f(y)$

 \item $f(x) \leq 0$ for all $x \in \mathbb{R}^n$ with equality if and only if $x = 0$
\end{itemize}

\noindent
\end{enumerate}

\noindent
Using the definition of convexity, try to show that a function satisfying the properties above is convex (the third condition is not needed to show this).



% !split
\subsection{Standard steepest descent}


Before we proceed, we would like to discuss the approach called the
\textbf{standard Steepest descent} (different from the above steepest descent discussion), which again leads to us having to be able
to compute a matrix. It belongs to the class of Conjugate Gradient methods (CG).

\href{{https://www.cs.cmu.edu/~quake-papers/painless-conjugate-gradient.pdf}}{The success of the CG method}
for finding solutions of non-linear problems is based on the theory
of conjugate gradients for linear systems of equations. It belongs to
the class of iterative methods for solving problems from linear
algebra of the type 
\begin{equation*} 
\bm{A}\bm{x} = \bm{b}.
\end{equation*} 

In the iterative process we end up with a problem like

\begin{equation*}
  \bm{r}= \bm{b}-\bm{A}\bm{x},
\end{equation*}
where $\bm{r}$ is the so-called residual or error in the iterative process.

When we have found the exact solution, $\bm{r}=0$.

% !split
\subsection{Gradient method}

The residual is zero when we reach the minimum of the quadratic equation
\begin{equation*}
  P(\bm{x})=\frac{1}{2}\bm{x}^T\bm{A}\bm{x} - \bm{x}^T\bm{b},
\end{equation*}

with the constraint that the matrix $\bm{A}$ is positive definite and
symmetric.  This defines also the Hessian and we want it to be  positive definite.  


% !split
\subsection{Steepest descent  method}

We denote the initial guess for $\bm{x}$ as $\bm{x}_0$. 
We can assume without loss of generality that
\begin{equation*}
\bm{x}_0=0,
\end{equation*}
or consider the system
\begin{equation*}
\bm{A}\bm{z} = \bm{b}-\bm{A}\bm{x}_0,
\end{equation*}
instead.


% !split
\subsection{Steepest descent  method}

% --- begin paragraph admon ---
\paragraph{}
One can show that the solution $\bm{x}$ is also the unique minimizer of the quadratic form
\begin{equation*}
  f(\bm{x}) = \frac{1}{2}\bm{x}^T\bm{A}\bm{x} - \bm{x}^T \bm{x} , \quad \bm{x}\in\mathbf{R}^n. 
\end{equation*}
This suggests taking the first basis vector $\bm{r}_1$ (see below for definition) 
to be the gradient of $f$ at $\bm{x}=\bm{x}_0$, 
which equals
\begin{equation*}
\bm{A}\bm{x}_0-\bm{b},
\end{equation*}
and 
$\bm{x}_0=0$ it is equal $-\bm{b}$.
% --- end paragraph admon ---



% !split
\subsection{Final expressions}

% --- begin paragraph admon ---
\paragraph{}
We can compute the residual iteratively as
\begin{equation*}
\bm{r}_{k+1}=\bm{b}-\bm{A}\bm{x}_{k+1},
 \end{equation*}
which equals
\begin{equation*}
\bm{b}-\bm{A}(\bm{x}_k+\alpha_k\bm{r}_k),
 \end{equation*}
or
\begin{equation*}
(\bm{b}-\bm{A}\bm{x}_k)-\alpha_k\bm{A}\bm{r}_k,
 \end{equation*}
which gives

\[
\alpha_k = \frac{\bm{r}_k^T\bm{r}_k}{\bm{r}_k^T\bm{A}\bm{r}_k}
\]
leading to the iterative scheme
\begin{equation*}
\bm{x}_{k+1}=\bm{x}_k-\alpha_k\bm{r}_{k},
 \end{equation*}
% --- end paragraph admon ---





% !split
\subsection{Steepest descent example}

\bpycod
import numpy as np
import numpy.linalg as la

import scipy.optimize as sopt

import matplotlib.pyplot as pt
from mpl_toolkits.mplot3d import axes3d

def f(x):
    return 0.5*x[0]**2 + 2.5*x[1]**2

def df(x):
    return np.array([x[0], 5*x[1]])

fig = pt.figure()
ax = fig.gca(projection="3d")

xmesh, ymesh = np.mgrid[-2:2:50j,-2:2:50j]
fmesh = f(np.array([xmesh, ymesh]))
ax.plot_surface(xmesh, ymesh, fmesh)
\epycod
And then as countor plot
\bpycod
pt.axis("equal")
pt.contour(xmesh, ymesh, fmesh)
guesses = [np.array([2, 2./5])]
\epycod
Find guesses
\bpycod
x = guesses[-1]
s = -df(x)
\epycod
Run it!
\bpycod
def f1d(alpha):
    return f(x + alpha*s)

alpha_opt = sopt.golden(f1d)
next_guess = x + alpha_opt * s
guesses.append(next_guess)
print(next_guess)
\epycod
What happened?
\bpycod
pt.axis("equal")
pt.contour(xmesh, ymesh, fmesh, 50)
it_array = np.array(guesses)
pt.plot(it_array.T[0], it_array.T[1], "x-")
\epycod

% !split
\subsection{Conjugate gradient method}

% --- begin paragraph admon ---
\paragraph{}
In the CG method we define so-called conjugate directions and two vectors 
$\bm{s}$ and $\bm{t}$
are said to be
conjugate if
\begin{equation*}
\bm{s}^T\bm{A}\bm{t}= 0.
\end{equation*}
The philosophy of the CG method is to perform searches in various conjugate directions
of our vectors $\bm{x}_i$ obeying the above criterion, namely
\begin{equation*}
\bm{x}_i^T\bm{A}\bm{x}_j= 0.
\end{equation*}
Two vectors are conjugate if they are orthogonal with respect to 
this inner product. Being conjugate is a symmetric relation: if $\bm{s}$ is conjugate to $\bm{t}$, then $\bm{t}$ is conjugate to $\bm{s}$.
% --- end paragraph admon ---



% !split
\subsection{Conjugate gradient method}

% --- begin paragraph admon ---
\paragraph{}
An example is given by the eigenvectors of the matrix
\begin{equation*}
\bm{v}_i^T\bm{A}\bm{v}_j= \lambda\bm{v}_i^T\bm{v}_j,
\end{equation*}
which is zero unless $i=j$.
% --- end paragraph admon ---




% !split
\subsection{Conjugate gradient method}

% --- begin paragraph admon ---
\paragraph{}
Assume now that we have a symmetric positive-definite matrix $\bm{A}$ of size
$n\times n$. At each iteration $i+1$ we obtain the conjugate direction of a vector
\begin{equation*}
\bm{x}_{i+1}=\bm{x}_{i}+\alpha_i\bm{p}_{i}. 
\end{equation*}
We assume that $\bm{p}_{i}$ is a sequence of $n$ mutually conjugate directions. 
Then the $\bm{p}_{i}$  form a basis of $R^n$ and we can expand the solution 
$  \bm{A}\bm{x} = \bm{b}$ in this basis, namely

\begin{equation*}
  \bm{x}  = \sum^{n}_{i=1} \alpha_i \bm{p}_i.
\end{equation*}
% --- end paragraph admon ---



% !split
\subsection{Conjugate gradient method}

% --- begin paragraph admon ---
\paragraph{}
The coefficients are given by
\begin{equation*}
    \mathbf{A}\mathbf{x} = \sum^{n}_{i=1} \alpha_i \mathbf{A} \mathbf{p}_i = \mathbf{b}.
\end{equation*}
Multiplying with $\bm{p}_k^T$  from the left gives

\begin{equation*}
  \bm{p}_k^T \bm{A}\bm{x} = \sum^{n}_{i=1} \alpha_i\bm{p}_k^T \bm{A}\bm{p}_i= \bm{p}_k^T \bm{b},
\end{equation*}
and we can define the coefficients $\alpha_k$ as

\begin{equation*}
    \alpha_k = \frac{\bm{p}_k^T \bm{b}}{\bm{p}_k^T \bm{A} \bm{p}_k}
\end{equation*}
% --- end paragraph admon ---



% !split
\subsection{Conjugate gradient method and iterations}

% --- begin paragraph admon ---
\paragraph{}

If we choose the conjugate vectors $\bm{p}_k$ carefully, 
then we may not need all of them to obtain a good approximation to the solution 
$\bm{x}$. 
We want to regard the conjugate gradient method as an iterative method. 
This will us to solve systems where $n$ is so large that the direct 
method would take too much time.

We denote the initial guess for $\bm{x}$ as $\bm{x}_0$. 
We can assume without loss of generality that
\begin{equation*}
\bm{x}_0=0,
\end{equation*}
or consider the system
\begin{equation*}
\bm{A}\bm{z} = \bm{b}-\bm{A}\bm{x}_0,
\end{equation*}
instead.
% --- end paragraph admon ---




% !split
\subsection{Conjugate gradient method}

% --- begin paragraph admon ---
\paragraph{}
One can show that the solution $\bm{x}$ is also the unique minimizer of the quadratic form
\begin{equation*}
  f(\bm{x}) = \frac{1}{2}\bm{x}^T\bm{A}\bm{x} - \bm{x}^T \bm{x} , \quad \bm{x}\in\mathbf{R}^n. 
\end{equation*}
This suggests taking the first basis vector $\bm{p}_1$ 
to be the gradient of $f$ at $\bm{x}=\bm{x}_0$, 
which equals
\begin{equation*}
\bm{A}\bm{x}_0-\bm{b},
\end{equation*}
and 
$\bm{x}_0=0$ it is equal $-\bm{b}$.
The other vectors in the basis will be conjugate to the gradient, 
hence the name conjugate gradient method.
% --- end paragraph admon ---




% !split
\subsection{Conjugate gradient method}

% --- begin paragraph admon ---
\paragraph{}
Let  $\bm{r}_k$ be the residual at the $k$-th step:
\begin{equation*}
\bm{r}_k=\bm{b}-\bm{A}\bm{x}_k.
\end{equation*}
Note that $\bm{r}_k$ is the negative gradient of $f$ at 
$\bm{x}=\bm{x}_k$, 
so the gradient descent method would be to move in the direction $\bm{r}_k$. 
Here, we insist that the directions $\bm{p}_k$ are conjugate to each other, 
so we take the direction closest to the gradient $\bm{r}_k$  
under the conjugacy constraint. 
This gives the following expression
\begin{equation*}
\bm{p}_{k+1}=\bm{r}_k-\frac{\bm{p}_k^T \bm{A}\bm{r}_k}{\bm{p}_k^T\bm{A}\bm{p}_k} \bm{p}_k.
\end{equation*}
% --- end paragraph admon ---



% !split
\subsection{Conjugate gradient method}

% --- begin paragraph admon ---
\paragraph{}
We can also  compute the residual iteratively as
\begin{equation*}
\bm{r}_{k+1}=\bm{b}-\bm{A}\bm{x}_{k+1},
 \end{equation*}
which equals
\begin{equation*}
\bm{b}-\bm{A}(\bm{x}_k+\alpha_k\bm{p}_k),
 \end{equation*}
or
\begin{equation*}
(\bm{b}-\bm{A}\bm{x}_k)-\alpha_k\bm{A}\bm{p}_k,
 \end{equation*}
which gives

\begin{equation*}
\bm{r}_{k+1}=\bm{r}_k-\bm{A}\bm{p}_{k},
 \end{equation*}
% --- end paragraph admon ---







% !split 
\subsection{Revisiting our first homework}

We will use linear regression as a case study for the gradient descent
methods. Linear regression is a great test case for the gradient
descent methods discussed in the lectures since it has several
desirable properties such as:

\begin{enumerate}
\item An analytical solution (recall homework set 1).

\item The gradient can be computed analytically.

\item The cost function is convex which guarantees that gradient descent converges for small enough learning rates
\end{enumerate}

\noindent
We revisit an example similar to what we had in the first homework set. We had a function  of the type

\bpycod
x = 2*np.random.rand(m,1)
y = 4+3*x+np.random.randn(m,1)
\epycod
with $x_i \in [0,1] $ is chosen randomly using a uniform distribution. Additionally we have a stochastic noise chosen according to a normal distribution $\cal {N}(0,1)$. 
The linear regression model is given by 
\[
h_\beta(x) = \bm{y} = \beta_0 + \beta_1 x,
\] 
such that 
\[
\bm{y}_i = \beta_0 + \beta_1 x_i.
\]

% !split 
\subsection{Gradient descent example}

Let $\mathbf{y} = (y_1,\cdots,y_n)^T$, $\mathbf{\bm{y}} = (\bm{y}_1,\cdots,\bm{y}_n)^T$ and $\beta = (\beta_0, \beta_1)^T$

It is convenient to write $\mathbf{\bm{y}} = X\beta$ where $X \in \mathbb{R}^{100 \times 2} $ is the design matrix given by (we keep the intercept here)
\[
X \equiv \begin{bmatrix}
1 & x_1  \\
\vdots & \vdots  \\
1 & x_{100} &  \\
\end{bmatrix}.
\]
The cost/loss/risk function is given by (
\[
C(\beta) = \frac{1}{n}||X\beta-\mathbf{y}||_{2}^{2} = \frac{1}{n}\sum_{i=1}^{100}\left[ (\beta_0 + \beta_1 x_i)^2 - 2 y_i (\beta_0 + \beta_1 x_i) + y_i^2\right] 
\]
and we want to find $\beta$ such that $C(\beta)$ is minimized.

% !split
\subsection{The derivative of the cost/loss function}

Computing $\partial C(\beta) / \partial \beta_0$ and $\partial C(\beta) / \partial \beta_1$ we can show  that the gradient can be written as
\[
\nabla_{\beta} C(\beta) = \frac{2}{n}\begin{bmatrix} \sum_{i=1}^{100} \left(\beta_0+\beta_1x_i-y_i\right) \\
\sum_{i=1}^{100}\left( x_i (\beta_0+\beta_1x_i)-y_ix_i\right) \\
\end{bmatrix} = \frac{2}{n}X^T(X\beta - \mathbf{y}), 
\]
where $X$ is the design matrix defined above.

% !split
\subsection{The Hessian matrix}
The Hessian matrix of $C(\beta)$ is given by 
\[
\bm{H} \equiv \begin{bmatrix}
\frac{\partial^2 C(\beta)}{\partial \beta_0^2} & \frac{\partial^2 C(\beta)}{\partial \beta_0 \partial \beta_1}  \\
\frac{\partial^2 C(\beta)}{\partial \beta_0 \partial \beta_1} & \frac{\partial^2 C(\beta)}{\partial \beta_1^2} &  \\
\end{bmatrix} = \frac{2}{n}X^T X.
\]
This result implies that $C(\beta)$ is a convex function since the matrix $X^T X$ always is positive semi-definite.




% !split
\subsection{Simple program}

We can now write a program that minimizes $C(\beta)$ using the gradient descent method with a constant learning rate $\gamma$ according to 
\[
\beta_{k+1} = \beta_k - \gamma \nabla_\beta C(\beta_k), \ k=0,1,\cdots 
\]

We can use the expression we computed for the gradient and let use a
$\beta_0$ be chosen randomly and let $\gamma = 0.001$. Stop iterating
when $||\nabla_\beta C(\beta_k) || \leq \epsilon = 10^{-8}$. \textbf{Note that the code below does not include the latter stop criterion}.

And finally we can compare our solution for $\beta$ with the analytic result given by 
$\beta= (X^TX)^{-1} X^T \mathbf{y}$.

% !split
\subsection{Gradient Descent Example}

Here our simple example
\bpycod

# Importing various packages
from random import random, seed
import numpy as np
import matplotlib.pyplot as plt
from mpl_toolkits.mplot3d import Axes3D
from matplotlib import cm
from matplotlib.ticker import LinearLocator, FormatStrFormatter
import sys

# the number of datapoints
n = 100
x = 2*np.random.rand(n,1)
y = 4+3*x+np.random.randn(n,1)

X = np.c_[np.ones((n,1)), x]
# Hessian matrix
H = (2.0/n)* X.T @ X
# Get the eigenvalues
EigValues, EigVectors = np.linalg.eig(H)
print(EigValues)

beta_linreg = np.linalg.inv(X.T @ X) @ X.T @ y
print(beta_linreg)
beta = np.random.randn(2,1)

eta = 1.0/np.max(EigValues)
Niterations = 1000

for iter in range(Niterations):
    gradient = (2.0/n)*X.T @ (X @ beta-y)
    beta -= eta*gradient

print(beta)
xnew = np.array([[0],[2]])
xbnew = np.c_[np.ones((2,1)), xnew]
ypredict = xbnew.dot(beta)
ypredict2 = xbnew.dot(beta_linreg)
plt.plot(xnew, ypredict, "r-")
plt.plot(xnew, ypredict2, "b-")
plt.plot(x, y ,'ro')
plt.axis([0,2.0,0, 15.0])
plt.xlabel(r'$x$')
plt.ylabel(r'$y$')
plt.title(r'Gradient descent example')
plt.show()

\epycod

% !split
\subsection{And a corresponding example using \textbf{scikit-learn}}

\bpycod
# Importing various packages
from random import random, seed
import numpy as np
import matplotlib.pyplot as plt
from sklearn.linear_model import SGDRegressor

n = 100
x = 2*np.random.rand(n,1)
y = 4+3*x+np.random.randn(n,1)

X = np.c_[np.ones((n,1)), x]
beta_linreg = np.linalg.inv(X.T @ X) @ (X.T @ y)
print(beta_linreg)
sgdreg = SGDRegressor(max_iter = 50, penalty=None, eta0=0.1)
sgdreg.fit(x,y.ravel())
print(sgdreg.intercept_, sgdreg.coef_)

\epycod



% !split 
\subsection{Gradient descent and Ridge}

We have also discussed Ridge regression where the loss function contains a regularized term given by the $L_2$ norm of $\beta$, 
\[
C_{\text{ridge}}(\beta) = \frac{1}{n}||X\beta -\mathbf{y}||^2 + \lambda ||\beta||^2, \ \lambda \geq 0.
\]

In order to minimize $C_{\text{ridge}}(\beta)$ using GD we only have adjust the gradient as follows 
\[
\nabla_\beta C_{\text{ridge}}(\beta)  = \frac{2}{n}\begin{bmatrix} \sum_{i=1}^{100} \left(\beta_0+\beta_1x_i-y_i\right) \\
\sum_{i=1}^{100}\left( x_i (\beta_0+\beta_1x_i)-y_ix_i\right) \\
\end{bmatrix} + 2\lambda\begin{bmatrix} \beta_0 \\ \beta_1\end{bmatrix} = 2 (X^T(X\beta - \mathbf{y})+\lambda \beta).
\]

We can easily extend our program to minimize $C_{\text{ridge}}(\beta)$ using gradient descent and compare with the analytical solution given by 
\[
\beta_{\text{ridge}} = \left(X^T X + \lambda I_{2 \times 2} \right)^{-1} X^T \mathbf{y}.
\]


% !split
\subsection{Program example for gradient descent with Ridge Regression}
\bpycod
from random import random, seed
import numpy as np
import matplotlib.pyplot as plt
from mpl_toolkits.mplot3d import Axes3D
from matplotlib import cm
from matplotlib.ticker import LinearLocator, FormatStrFormatter
import sys

# the number of datapoints
n = 100
x = 2*np.random.rand(n,1)
y = 4+3*x+np.random.randn(n,1)

X = np.c_[np.ones((n,1)), x]
XT_X = X.T @ X

#Ridge parameter lambda
lmbda  = 0.001
Id = lmbda* np.eye(XT_X.shape[0])

beta_linreg = np.linalg.inv(XT_X+Id) @ X.T @ y
print(beta_linreg)
# Start plain gradient descent
beta = np.random.randn(2,1)

eta = 0.1
Niterations = 100

for iter in range(Niterations):
    gradients = 2.0/n*X.T @ (X @ (beta)-y)+2*lmbda*beta
    beta -= eta*gradients

print(beta)
ypredict = X @ beta
ypredict2 = X @ beta_linreg
plt.plot(x, ypredict, "r-")
plt.plot(x, ypredict2, "b-")
plt.plot(x, y ,'ro')
plt.axis([0,2.0,0, 15.0])
plt.xlabel(r'$x$')
plt.ylabel(r'$y$')
plt.title(r'Gradient descent example for Ridge')
plt.show()


\epycod

% !split
\subsection{Using gradient descent methods, limitations}

\begin{itemize}
\item \textbf{Gradient descent (GD) finds local minima of our function}. Since the GD algorithm is deterministic, if it converges, it will converge to a local minimum of our cost/loss/risk function. Because in ML we are often dealing with extremely rugged landscapes with many local minima, this can lead to poor performance.

\item \textbf{GD is sensitive to initial conditions}. One consequence of the local nature of GD is that initial conditions matter. Depending on where one starts, one will end up at a different local minima. Therefore, it is very important to think about how one initializes the training process. This is true for GD as well as more complicated variants of GD.

\item \textbf{Gradients are computationally expensive to calculate for large datasets}. In many cases in statistics and ML, the cost/loss/risk function is a sum of terms, with one term for each data point. For example, in linear regression, $E \propto \sum_{i=1}^n (y_i - \mathbf{w}^T\cdot\mathbf{x}_i)^2$; for logistic regression, the square error is replaced by the cross entropy. To calculate the gradient we have to sum over \emph{all} $n$ data points. Doing this at every GD step becomes extremely computationally expensive. An ingenious solution to this, is to calculate the gradients using small subsets of the data called ``mini batches''. This has the added benefit of introducing stochasticity into our algorithm.

\item \textbf{GD is very sensitive to choices of learning rates}. GD is extremely sensitive to the choice of learning rates. If the learning rate is very small, the training process take an extremely long time. For larger learning rates, GD can diverge and give poor results. Furthermore, depending on what the local landscape looks like, we have to modify the learning rates to ensure convergence. Ideally, we would \emph{adaptively} choose the learning rates to match the landscape.

\item \textbf{GD treats all directions in parameter space uniformly.} Another major drawback of GD is that unlike Newton's method, the learning rate for GD is the same in all directions in parameter space. For this reason, the maximum learning rate is set by the behavior of the steepest direction and this can significantly slow down training. Ideally, we would like to take large steps in flat directions and small steps in steep directions. Since we are exploring rugged landscapes where curvatures change, this requires us to keep track of not only the gradient but second derivatives. The ideal scenario would be to calculate the Hessian but this proves to be too computationally expensive. 

\item GD can take exponential time to escape saddle points, even with random initialization. As we mentioned, GD is extremely sensitive to initial condition since it determines the particular local minimum GD would eventually reach. However, even with a good initialization scheme, through the introduction of randomness, GD can still take exponential time to escape saddle points.
\end{itemize}

\noindent
% !split
\subsection{Stochastic Gradient Descent}

Stochastic gradient descent (SGD) and variants thereof address some of
the shortcomings of the Gradient descent method discussed above.

The underlying idea of SGD comes from the observation that the cost
function, which we want to minimize, can almost always be written as a
sum over $n$ data points $\{\mathbf{x}_i\}_{i=1}^n$,
\[
C(\mathbf{\beta}) = \sum_{i=1}^n c_i(\mathbf{x}_i,
\mathbf{\beta}). 
\]

% !split
\subsection{Computation of gradients}

This in turn means that the gradient can be
computed as a sum over $i$-gradients 
\[
\nabla_\beta C(\mathbf{\beta}) = \sum_i^n \nabla_\beta c_i(\mathbf{x}_i,
\mathbf{\beta}).
\]

Stochasticity/randomness is introduced by only taking the
gradient on a subset of the data called minibatches.  If there are $n$
data points and the size of each minibatch is $M$, there will be $n/M$
minibatches. We denote these minibatches by $B_k$ where
$k=1,\cdots,n/M$.

% !split
\subsection{SGD example}
As an example, suppose we have $10$ data points $(\mathbf{x}_1,\cdots, \mathbf{x}_{10})$ 
and we choose to have $M=5$ minibathces,
then each minibatch contains two data points. In particular we have
$B_1 = (\mathbf{x}_1,\mathbf{x}_2), \cdots, B_5 =
(\mathbf{x}_9,\mathbf{x}_{10})$. Note that if you choose $M=1$ you
have only a single batch with all data points and on the other extreme,
you may choose $M=n$ resulting in a minibatch for each datapoint, i.e
$B_k = \mathbf{x}_k$.

The idea is now to approximate the gradient by replacing the sum over
all data points with a sum over the data points in one the minibatches
picked at random in each gradient descent step 
\[
\nabla_{\beta}
C(\mathbf{\beta}) = \sum_{i=1}^n \nabla_\beta c_i(\mathbf{x}_i,
\mathbf{\beta}) \rightarrow \sum_{i \in B_k}^n \nabla_\beta
c_i(\mathbf{x}_i, \mathbf{\beta}).
\]

% !split
\subsection{The gradient step}

Thus a gradient descent step now looks like 
\[
\beta_{j+1} = \beta_j - \gamma_j \sum_{i \in B_k}^n \nabla_\beta c_i(\mathbf{x}_i,
\mathbf{\beta})
\]

where $k$ is picked at random with equal
probability from $[1,n/M]$. An iteration over the number of
minibathces (n/M) is commonly referred to as an epoch. Thus it is
typical to choose a number of epochs and for each epoch iterate over
the number of minibatches, as exemplified in the code below.

% !split
\subsection{Simple example code}

\bpycod
import numpy as np 

n = 100 #100 datapoints 
M = 5   #size of each minibatch
m = int(n/M) #number of minibatches
n_epochs = 10 #number of epochs

j = 0
for epoch in range(1,n_epochs+1):
    for i in range(m):
        k = np.random.randint(m) #Pick the k-th minibatch at random
        #Compute the gradient using the data in minibatch Bk
        #Compute new suggestion for 
        j += 1
\epycod

Taking the gradient only on a subset of the data has two important
benefits. First, it introduces randomness which decreases the chance
that our opmization scheme gets stuck in a local minima. Second, if
the size of the minibatches are small relative to the number of
datapoints ($M <  n$), the computation of the gradient is much
cheaper since we sum over the datapoints in the $k-th$ minibatch and not
all $n$ datapoints.

% !split
\subsection{When do we stop?}

A natural question is when do we stop the search for a new minimum?
One possibility is to compute the full gradient after a given number
of epochs and check if the norm of the gradient is smaller than some
threshold and stop if true. However, the condition that the gradient
is zero is valid also for local minima, so this would only tell us
that we are close to a local/global minimum. However, we could also
evaluate the cost function at this point, store the result and
continue the search. If the test kicks in at a later stage we can
compare the values of the cost function and keep the $\beta$ that
gave the lowest value.

% !split
\subsection{Slightly different approach}

Another approach is to let the step length $\gamma_j$ depend on the
number of epochs in such a way that it becomes very small after a
reasonable time such that we do not move at all.

As an example, let $e = 0,1,2,3,\cdots$ denote the current epoch and let $t_0, t_1 > 0$ be two fixed numbers. Furthermore, let $t = e \cdot m + i$ where $m$ is the number of minibatches and $i=0,\cdots,m-1$. Then the function $$\gamma_j(t; t_0, t_1) = \frac{t_0}{t+t_1} $$ goes to zero as the number of epochs gets large. I.e. we start with a step length $\gamma_j (0; t_0, t_1) = t_0/t_1$ which decays in \emph{time} $t$.

In this way we can fix the number of epochs, compute $\beta$ and
evaluate the cost function at the end. Repeating the computation will
give a different result since the scheme is random by design. Then we
pick the final $\beta$ that gives the lowest value of the cost
function.

\bpycod
import numpy as np 

def step_length(t,t0,t1):
    return t0/(t+t1)

n = 100 #100 datapoints 
M = 5   #size of each minibatch
m = int(n/M) #number of minibatches
n_epochs = 500 #number of epochs
t0 = 1.0
t1 = 10

gamma_j = t0/t1
j = 0
for epoch in range(1,n_epochs+1):
    for i in range(m):
        k = np.random.randint(m) #Pick the k-th minibatch at random
        #Compute the gradient using the data in minibatch Bk
        #Compute new suggestion for beta
        t = epoch*m+i
        gamma_j = step_length(t,t0,t1)
        j += 1

print("gamma_j after %d epochs: %g" % (n_epochs,gamma_j))
\epycod





% !split
\subsection{Program for stochastic gradient}

\bpycod
# Importing various packages
from math import exp, sqrt
from random import random, seed
import numpy as np
import matplotlib.pyplot as plt
from sklearn.linear_model import SGDRegressor

m = 100
x = 2*np.random.rand(m,1)
y = 4+3*x+np.random.randn(m,1)

X = np.c_[np.ones((m,1)), x]
theta_linreg = np.linalg.inv(X.T @ X) @ (X.T @ y)
print("Own inversion")
print(theta_linreg)
sgdreg = SGDRegressor(max_iter = 50, penalty=None, eta0=0.1)
sgdreg.fit(x,y.ravel())
print("sgdreg from scikit")
print(sgdreg.intercept_, sgdreg.coef_)


theta = np.random.randn(2,1)
eta = 0.1
Niterations = 1000


for iter in range(Niterations):
    gradients = 2.0/m*X.T @ ((X @ theta)-y)
    theta -= eta*gradients
print("theta from own gd")
print(theta)

xnew = np.array([[0],[2]])
Xnew = np.c_[np.ones((2,1)), xnew]
ypredict = Xnew.dot(theta)
ypredict2 = Xnew.dot(theta_linreg)


n_epochs = 50
t0, t1 = 5, 50
def learning_schedule(t):
    return t0/(t+t1)

theta = np.random.randn(2,1)

for epoch in range(n_epochs):
    for i in range(m):
        random_index = np.random.randint(m)
        xi = X[random_index:random_index+1]
        yi = y[random_index:random_index+1]
        gradients = 2 * xi.T @ ((xi @ theta)-yi)
        eta = learning_schedule(epoch*m+i)
        theta = theta - eta*gradients
print("theta from own sdg")
print(theta)

plt.plot(xnew, ypredict, "r-")
plt.plot(xnew, ypredict2, "b-")
plt.plot(x, y ,'ro')
plt.axis([0,2.0,0, 15.0])
plt.xlabel(r'$x$')
plt.ylabel(r'$y$')
plt.title(r'Random numbers ')
plt.show()

\epycod

\textbf{Challenge}: try to write a similar code for a Logistic Regression case.









% !split
\subsection{Momentum based GD}

The stochastic gradient descent (SGD) is almost always used with a
\emph{momentum} or inertia term that serves as a memory of the direction we
are moving in parameter space.  This is typically implemented as
follows

\begin{align}
\mathbf{v}_{t}&=\gamma \mathbf{v}_{t-1}+\eta_{t}\nabla_\theta E(\boldsymbol{\theta}_t) \nonumber \\
\boldsymbol{\theta}_{t+1}&= \boldsymbol{\theta}_t -\mathbf{v}_{t},
\end{align}

where we have introduced a momentum parameter $\gamma$, with
$0\le\gamma\le 1$, and for brevity we dropped the explicit notation to
indicate the gradient is to be taken over a different mini-batch at
each step. We call this algorithm gradient descent with momentum
(GDM). From these equations, it is clear that $\mathbf{v}_t$ is a
running average of recently encountered gradients and
$(1-\gamma)^{-1}$ sets the characteristic time scale for the memory
used in the averaging procedure. Consistent with this, when
$\gamma=0$, this just reduces down to ordinary SGD as discussed
earlier. An equivalent way of writing the updates is

\[
\Delta \boldsymbol{\theta}_{t+1} = \gamma \Delta \boldsymbol{\theta}_t -\ \eta_{t}\nabla_\theta E(\boldsymbol{\theta}_t),
\]
where we have defined $\Delta \boldsymbol{\theta}_{t}= \boldsymbol{\theta}_t-\boldsymbol{\theta}_{t-1}$.

% !split
\subsection{More on momentum based approaches}

Let us try to get more intuition from these equations. It is helpful
to consider a simple physical analogy with a particle of mass $m$
moving in a viscous medium with drag coefficient $\mu$ and potential
$E(\mathbf{w})$. If we denote the particle's position by $\mathbf{w}$,
then its motion is described by

\[
m {d^2 \mathbf{w} \over dt^2} + \mu {d \mathbf{w} \over dt }= -\nabla_w E(\mathbf{w}).
\]

We can discretize this equation in the usual way to get

\[
m { \mathbf{w}_{t+\Delta t}-2 \mathbf{w}_{t} +\mathbf{w}_{t-\Delta t} \over (\Delta t)^2}+\mu {\mathbf{w}_{t+\Delta t}- \mathbf{w}_{t} \over \Delta t} = -\nabla_w E(\mathbf{w}).
\]

Rearranging this equation, we can rewrite this as

\[
\Delta \mathbf{w}_{t +\Delta t}= - { (\Delta t)^2 \over m +\mu \Delta t} \nabla_w E(\mathbf{w})+ {m \over m +\mu \Delta t} \Delta \mathbf{w}_t.
\]

% !split
\subsection{Momentum parameter}

Notice that this equation is identical to previous one if we identify
the position of the particle, $\mathbf{w}$, with the parameters
$\boldsymbol{\theta}$. This allows us to identify the momentum
parameter and learning rate with the mass of the particle and the
viscous drag as:

\[
\gamma= {m \over m +\mu \Delta t }, \qquad \eta = {(\Delta t)^2 \over m +\mu \Delta t}.
\]

Thus, as the name suggests, the momentum parameter is proportional to
the mass of the particle and effectively provides inertia.
Furthermore, in the large viscosity/small learning rate limit, our
memory time scales as $(1-\gamma)^{-1} \approx m/(\mu \Delta t)$.

Why is momentum useful? SGD momentum helps the gradient descent
algorithm gain speed in directions with persistent but small gradients
even in the presence of stochasticity, while suppressing oscillations
in high-curvature directions. This becomes especially important in
situations where the landscape is shallow and flat in some directions
and narrow and steep in others. It has been argued that first-order
methods (with appropriate initial conditions) can perform comparable
to more expensive second order methods, especially in the context of
complex deep learning models.

These beneficial properties of momentum can sometimes become even more
pronounced by using a slight modification of the classical momentum
algorithm called Nesterov Accelerated Gradient (NAG).

In the NAG algorithm, rather than calculating the gradient at the
current parameters, $\nabla_\theta E(\boldsymbol{\theta}_t)$, one
calculates the gradient at the expected value of the parameters given
our current momentum, $\nabla_\theta E(\boldsymbol{\theta}_t +\gamma
\mathbf{v}_{t-1})$. This yields the NAG update rule

\begin{align}
\mathbf{v}_{t}&=\gamma \mathbf{v}_{t-1}+\eta_{t}\nabla_\theta E(\boldsymbol{\theta}_t +\gamma \mathbf{v}_{t-1}) \nonumber \\
\boldsymbol{\theta}_{t+1}&= \boldsymbol{\theta}_t -\mathbf{v}_{t}.
\end{align}

One of the major advantages of NAG is that it allows for the use of a larger learning rate than GDM for the same choice of $\gamma$.


% !split
\subsection{Second moment of the gradient}


In stochastic gradient descent, with and without momentum, we still
have to specify a schedule for tuning the learning rates $\eta_t$
as a function of time.  As discussed in the context of Newton's
method, this presents a number of dilemmas. The learning rate is
limited by the steepest direction which can change depending on the
current position in the landscape. To circumvent this problem, ideally
our algorithm would keep track of curvature and take large steps in
shallow, flat directions and small steps in steep, narrow directions.
Second-order methods accomplish this by calculating or approximating
the Hessian and normalizing the learning rate by the
curvature. However, this is very computationally expensive for
extremely large models. Ideally, we would like to be able to
adaptively change the step size to match the landscape without paying
the steep computational price of calculating or approximating
Hessians.

Recently, a number of methods have been introduced that accomplish
this by tracking not only the gradient, but also the second moment of
the gradient. These methods include AdaGrad, AdaDelta, RMS-Prop, and
ADAM.

% !split
\subsection{RMS prop}

In RMS prop, in addition to keeping a running average of the first
moment of the gradient, we also keep track of the second moment
denoted by $\mathbf{s}_t=\mathbb{E}[\mathbf{g}_t^2]$. The update rule
for RMS prop is given by

\begin{align}
\mathbf{g}_t &= \nabla_\theta E(\boldsymbol{\theta}) \\
\mathbf{s}_t &=\beta \mathbf{s}_{t-1} +(1-\beta)\mathbf{g}_t^2 \nonumber \\
\boldsymbol{\theta}_{t+1}&=&\boldsymbol{\theta}_t - \eta_t { \mathbf{g}_t \over \sqrt{\mathbf{s}_t +\epsilon}}, \nonumber
\end{align}

where $\beta$ controls the averaging time of the second moment and is
typically taken to be about $\beta=0.9$, $\eta_t$ is a learning rate
typically chosen to be $10^{-3}$, and $\epsilon\sim 10^{-8} $ is a
small regularization constant to prevent divergences. Multiplication
and division by vectors is understood as an element-wise operation. It
is clear from this formula that the learning rate is reduced in
directions where the norm of the gradient is consistently large. This
greatly speeds up the convergence by allowing us to use a larger
learning rate for flat directions.


% !split
\subsection{ADAM optimizer}

A related algorithm is the ADAM optimizer. In ADAM, we keep a running
average of both the first and second moment of the gradient and use
this information to adaptively change the learning rate for different
parameters. In addition to keeping a running average of the first and
second moments of the gradient
(i.e.~$\mathbf{m}_t=\mathbb{E}[\mathbf{g}_t]$ and
$\mathbf{s}_t=\mathbb{E}[\mathbf{g}^2_t]$, respectively), ADAM
performs an additional bias correction to account for the fact that we
are estimating the first two moments of the gradient using a running
average (denoted by the hats in the update rule below). The update
rule for ADAM is given by (where multiplication and division are once
again understood to be element-wise operations below)

\begin{align}
\mathbf{g}_t &= \nabla_\theta E(\boldsymbol{\theta}) \\
\mathbf{m}_t &= \beta_1 \mathbf{m}_{t-1} + (1-\beta_1) \mathbf{g}_t \nonumber \\
\mathbf{s}_t &=\beta_2 \mathbf{s}_{t-1} +(1-\beta_2)\mathbf{g}_t^2 \nonumber \\
\bm{\mathbf{m}}_t&={\mathbf{m}_t \over 1-\beta_1^t} \nonumber \\
\bm{\mathbf{s}}_t &={\mathbf{s}_t \over1-\beta_2^t} \nonumber \\
\boldsymbol{\theta}_{t+1}&=\boldsymbol{\theta}_t - \eta_t { \bm{\mathbf{m}}_t \over \sqrt{\bm{\mathbf{s}}_t} +\epsilon}, \nonumber \\
\end{align}

where $\beta_1$ and $\beta_2$ set the memory lifetime of the first and
second moment and are typically taken to be $0.9$ and $0.99$
respectively, and $\eta$ and $\epsilon$ are identical to RMSprop.

Like in RMSprop, the effective step size of a parameter depends on the
magnitude of its gradient squared.  To understand this better, let us
rewrite this expression in terms of the variance
$\boldsymbol{\sigma}_t^2 = \bm{\mathbf{s}}_t -
(\bm{\mathbf{m}}_t)^2$. Consider a single parameter $\theta_t$. The
update rule for this parameter is given by

\[
\Delta \theta_{t+1}= -\eta_t { \bm{m}_t \over \sqrt{\sigma_t^2 +  m_t^2 }+\epsilon}.
\]




% !split
\subsection{Practical tips}

\begin{itemize}
\item \textbf{Randomize the data when making mini-batches}. It is always important to randomly shuffle the data when forming mini-batches. Otherwise, the gradient descent method can fit spurious correlations resulting from the order in which data is presented.

\item \textbf{Transform your inputs}. Learning becomes difficult when our landscape has a mixture of steep and flat directions. One simple trick for minimizing these situations is to standardize the data by subtracting the mean and normalizing the variance of input variables. Whenever possible, also decorrelate the inputs. To understand why this is helpful, consider the case of linear regression. It is easy to show that for the squared error cost function, the Hessian of the cost function is just the correlation matrix between the inputs. Thus, by standardizing the inputs, we are ensuring that the landscape looks homogeneous in all directions in parameter space. Since most deep networks can be viewed as linear transformations followed by a non-linearity at each layer, we expect this intuition to hold beyond the linear case.

\item \textbf{Monitor the out-of-sample performance.} Always monitor the performance of your model on a validation set (a small portion of the training data that is held out of the training process to serve as a proxy for the test set. If the validation error starts increasing, then the model is beginning to overfit. Terminate the learning process. This \emph{early stopping} significantly improves performance in many settings.

\item \textbf{Adaptive optimization methods don't always have good generalization.} Recent studies have shown that adaptive methods such as ADAM, RMSPorp, and AdaGrad tend to have poor generalization compared to SGD or SGD with momentum, particularly in the high-dimensional limit (i.e.~the number of parameters exceeds the number of data points). Although it is not clear at this stage why these methods perform so well in training deep neural networks, simpler procedures like properly-tuned SGD may work as well or better in these applications.
\end{itemize}

\noindent
Geron's text, see chapter 11, has several interesting discussions.



% !split
\subsection{Automatic differentiation}

\href{{https://en.wikipedia.org/wiki/Automatic_differentiation}}{Automatic differentiation (AD)}, 
also called algorithmic
differentiation or computational differentiation,is a set of
techniques to numerically evaluate the derivative of a function
specified by a computer program. AD exploits the fact that every
computer program, no matter how complicated, executes a sequence of
elementary arithmetic operations (addition, subtraction,
multiplication, division, etc.) and elementary functions (exp, log,
sin, cos, etc.). By applying the chain rule repeatedly to these
operations, derivatives of arbitrary order can be computed
automatically, accurately to working precision, and using at most a
small constant factor more arithmetic operations than the original
program.

Automatic differentiation is neither:

\begin{itemize}
\item Symbolic differentiation, nor

\item Numerical differentiation (the method of finite differences).
\end{itemize}

\noindent
Symbolic differentiation can lead to inefficient code and faces the
difficulty of converting a computer program into a single expression,
while numerical differentiation can introduce round-off errors in the
discretization process and cancellation



Python has tools for so-called \textbf{automatic differentiation}.
Consider the following example
\[
f(x) = \sin\left(2\pi x + x^2\right)
\]
which has the following derivative
\[
f'(x) = \cos\left(2\pi x + x^2\right)\left(2\pi + 2x\right) 
\]
Using \textbf{autograd} we have

\bpycod
import autograd.numpy as np

# To do elementwise differentiation:
from autograd import elementwise_grad as egrad 

# To plot:
import matplotlib.pyplot as plt 


def f(x):
    return np.sin(2*np.pi*x + x**2)

def f_grad_analytic(x):
    return np.cos(2*np.pi*x + x**2)*(2*np.pi + 2*x)

# Do the comparison:
x = np.linspace(0,1,1000)

f_grad = egrad(f)

computed = f_grad(x)
analytic = f_grad_analytic(x)

plt.title('Derivative computed from Autograd compared with the analytical derivative')
plt.plot(x,computed,label='autograd')
plt.plot(x,analytic,label='analytic')

plt.xlabel('x')
plt.ylabel('y')
plt.legend()

plt.show()

print("The max absolute difference is: %g"%(np.max(np.abs(computed - analytic))))
\epycod

% !split 
\subsection{Using autograd}

Here we
experiment with what kind of functions Autograd is capable
of finding the gradient of. The following Python functions are just
meant to illustrate what Autograd can do, but please feel free to
experiment with other, possibly more complicated, functions as well.

\bpycod
import autograd.numpy as np
from autograd import grad

def f1(x):
    return x**3 + 1

f1_grad = grad(f1)

# Remember to send in float as argument to the computed gradient from Autograd!
a = 1.0

# See the evaluated gradient at a using autograd:
print("The gradient of f1 evaluated at a = %g using autograd is: %g"%(a,f1_grad(a)))

# Compare with the analytical derivative, that is f1'(x) = 3*x**2 
grad_analytical = 3*a**2
print("The gradient of f1 evaluated at a = %g by finding the analytic expression is: %g"%(a,grad_analytical))
\epycod


% !split
\subsection{Autograd with more complicated functions}

To differentiate with respect to two (or more) arguments of a Python
function, Autograd need to know at which variable the function if
being differentiated with respect to.

\bpycod
import autograd.numpy as np
from autograd import grad
def f2(x1,x2):
    return 3*x1**3 + x2*(x1 - 5) + 1

# By sending the argument 0, Autograd will compute the derivative w.r.t the first variable, in this case x1
f2_grad_x1 = grad(f2,0)

# ... and differentiate w.r.t x2 by sending 1 as an additional arugment to grad
f2_grad_x2 = grad(f2,1)

x1 = 1.0
x2 = 3.0 

print("Evaluating at x1 = %g, x2 = %g"%(x1,x2))
print("-"*30)

# Compare with the analytical derivatives:

# Derivative of f2 w.r.t x1 is: 9*x1**2 + x2:
f2_grad_x1_analytical = 9*x1**2 + x2

# Derivative of f2 w.r.t x2 is: x1 - 5:
f2_grad_x2_analytical = x1 - 5

# See the evaluated derivations:
print("The derivative of f2 w.r.t x1: %g"%( f2_grad_x1(x1,x2) ))
print("The analytical derivative of f2 w.r.t x1: %g"%( f2_grad_x1(x1,x2) ))

print()

print("The derivative of f2 w.r.t x2: %g"%( f2_grad_x2(x1,x2) ))
print("The analytical derivative of f2 w.r.t x2: %g"%( f2_grad_x2(x1,x2) ))
\epycod

Note that the grad function will not produce the true gradient of the function. The true gradient of a function with two or more variables will produce a vector, where each element is the function differentiated w.r.t a variable.


% !split
\subsection{More complicated functions using the elements of their arguments directly}

\bpycod
import autograd.numpy as np
from autograd import grad
def f3(x): # Assumes x is an array of length 5 or higher
    return 2*x[0] + 3*x[1] + 5*x[2] + 7*x[3] + 11*x[4]**2

f3_grad = grad(f3)

x = np.linspace(0,4,5)

# Print the computed gradient:
print("The computed gradient of f3 is: ", f3_grad(x))

# The analytical gradient is: (2, 3, 5, 7, 22*x[4])
f3_grad_analytical = np.array([2, 3, 5, 7, 22*x[4]])

# Print the analytical gradient:
print("The analytical gradient of f3 is: ", f3_grad_analytical)
\epycod

Note that in this case, when sending an array as input argument, the
output from Autograd is another array. This is the true gradient of
the function, as opposed to the function in the previous example. By
using arrays to represent the variables, the output from Autograd
might be easier to work with, as the output is closer to what one
could expect form a gradient-evaluting function.

% !split 
\subsection{Functions using mathematical functions from Numpy}

\bpycod
import autograd.numpy as np
from autograd import grad
def f4(x):
    return np.sqrt(1+x**2) + np.exp(x) + np.sin(2*np.pi*x)

f4_grad = grad(f4)

x = 2.7

# Print the computed derivative:
print("The computed derivative of f4 at x = %g is: %g"%(x,f4_grad(x)))

# The analytical derivative is: x/sqrt(1 + x**2) + exp(x) + cos(2*pi*x)*2*pi
f4_grad_analytical = x/np.sqrt(1 + x**2) + np.exp(x) + np.cos(2*np.pi*x)*2*np.pi

# Print the analytical gradient:
print("The analytical gradient of f4 at x = %g is: %g"%(x,f4_grad_analytical))
\epycod


% !split
\subsection{More autograd}

\bpycod
import autograd.numpy as np
from autograd import grad
def f5(x):
    if x >= 0:
        return x**2
    else:
        return -3*x + 1

f5_grad = grad(f5)

x = 2.7

# Print the computed derivative:
print("The computed derivative of f5 at x = %g is: %g"%(x,f5_grad(x)))
\epycod


% !split
\subsection{And  with loops}

\bpycod
import autograd.numpy as np
from autograd import grad
def f6_for(x):
    val = 0
    for i in range(10):
        val = val + x**i
    return val

def f6_while(x):
    val = 0
    i = 0
    while i < 10:
        val = val + x**i
        i = i + 1
    return val

f6_for_grad = grad(f6_for)
f6_while_grad = grad(f6_while)

x = 0.5

# Print the computed derivaties of f6_for and f6_while
print("The computed derivative of f6_for at x = %g is: %g"%(x,f6_for_grad(x)))
print("The computed derivative of f6_while at x = %g is: %g"%(x,f6_while_grad(x)))
\epycod
\bpycod
import autograd.numpy as np
from autograd import grad
# Both of the functions are implementation of the sum: sum(x**i) for i = 0, ..., 9
# The analytical derivative is: sum(i*x**(i-1)) 
f6_grad_analytical = 0
for i in range(10):
    f6_grad_analytical += i*x**(i-1)

print("The analytical derivative of f6 at x = %g is: %g"%(x,f6_grad_analytical))
\epycod

% !split
\subsection{Using recursion}
\bpycod
import autograd.numpy as np
from autograd import grad

def f7(n): # Assume that n is an integer
    if n == 1 or n == 0:
        return 1
    else:
        return n*f7(n-1)

f7_grad = grad(f7)

n = 2.0

print("The computed derivative of f7 at n = %d is: %g"%(n,f7_grad(n)))

# The function f7 is an implementation of the factorial of n.
# By using the product rule, one can find that the derivative is:

f7_grad_analytical = 0
for i in range(int(n)-1):
    tmp = 1
    for k in range(int(n)-1):
        if k != i:
            tmp *= (n - k)
    f7_grad_analytical += tmp

print("The analytical derivative of f7 at n = %d is: %g"%(n,f7_grad_analytical))

\epycod
Note that if n is equal to zero or one, Autograd will give an error message. This message appears when the output is independent on input.

% !split
\subsection{Unsupported functions}
Autograd supports many features. However, there are some functions that is not supported (yet) by Autograd.

Assigning a value to the variable being differentiated with respect to
\bpycod
import autograd.numpy as np
from autograd import grad
def f8(x): # Assume x is an array
    x[2] = 3
    return x*2

f8_grad = grad(f8)

x = 8.4

print("The derivative of f8 is:",f8_grad(x))
\epycod
Here, Autograd tells us that an 'ArrayBox' does not support item assignment. The item assignment is done when the program tries to assign x[2] to the value 3. However, Autograd has implemented the computation of the derivative such that this assignment is not possible.

% !split
\subsection{The syntax a.dot(b) when finding the dot product}
\bpycod
import autograd.numpy as np
from autograd import grad
def f9(a): # Assume a is an array with 2 elements
    b = np.array([1.0,2.0])
    return a.dot(b)

f9_grad = grad(f9)

x = np.array([1.0,0.0])

print("The derivative of f9 is:",f9_grad(x))
\epycod

Here we are told that the 'dot' function does not belong to Autograd's
version of a Numpy array.  To overcome this, an alternative syntax
which also computed the dot product can be used:

\bpycod
import autograd.numpy as np
from autograd import grad
def f9_alternative(x): # Assume a is an array with 2 elements
    b = np.array([1.0,2.0])
    return np.dot(x,b) # The same as x_1*b_1 + x_2*b_2

f9_alternative_grad = grad(f9_alternative)

x = np.array([3.0,0.0])

print("The gradient of f9 is:",f9_alternative_grad(x))

# The analytical gradient of the dot product of vectors x and b with two elements (x_1,x_2) and (b_1, b_2) respectively
# w.r.t x is (b_1, b_2).
\epycod

% !split
\subsection{Recommended to avoid}
The documentation recommends to avoid inplace operations such as
\bpycod
a += b
a -= b
a*= b
a /=b
\epycod




% ------------------- end of main content ---------------

% #ifdef PREAMBLE
\end{document}
% #endif

